%%%%%%%%%%%%%%%%%%%%%%%%%%%%%%%%%%%%%%%%%%%%%%%%%%%%%%%%%%%%%%%%%%%%%%
%
%.IDENTIFICATION $Id: template.tex.src,v 1.41 2008/01/25 10:47:12 fsogni Exp $
%.LANGUAGE       TeX, LaTeX
%.ENVIRONMENT    ESOFORM
%.PURPOSE        Template application form for ESO Observing time.
%.AUTHOR         The Esoform Package is maintained by the Observing
%                Programmes Office (OPO) while the background software
%                is provided by the User Support System (USS) Department.
%
%-----------------------------------------------------------------------
%
%
%                   ESO LA SILLA PARANAL OBSERVATORY
%                   --------------------------------
%                   NORMAL PROGRAMME PHASE 1 TEMPLATE
%                   ---------------------------------
%
%
%
%          PLEASE CHECK THE ESOFORM USERS' MANUAL FOR DETAILED 
%              INFORMATION AND DESCRIPTIONS OF THE MACROS. 
%     (see the file usersmanual.tex provided in the ESOFORM package) 
%
%
%        ====>>>> TO BE SUBMITTED THROUGH WEB UPLOAD  <<<<====
%               (see the Call for Proposals for details)
%
%%%%%%%%%%%%%%%%%%%%%%%%%%%%%%%%%%%%%%%%%%%%%%%%%%%%%%%%%%%%%%%%%%%%%%

%%%%%%%%%%%%%%%%%%%%%%%%%%%%%%%%%%%%%%%%%%%%%%%%%%%%%%%%%%%%%%%%%%%%%%
%
%                      I M P O R T A N T    N O T E
%                      ----------------------------
%
% By submitting this proposal, the Principal Investigator takes full
% responsibility for the content of the proposal, in particular with
% regard to the names of CoI's and the agreement to act in accordance
% with the ESO policy and regulations, should observing time be
% granted.
%
%%%%%%%%%%%%%%%%%%%%%%%%%%%%%%%%%%%%%%%%%%%%%%%%%%%%%%%%%%%%%%%%%%%%%% 

%
%    - LaTeX *is* sensitive towards upper and lower case letters.
%    - Everything after a `%' character is taken as comments.
%    - DO NOT CHANGE ANY OF THE MACRO NAMES (words beginning with `\')
%    - DO NOT INSERT ANY TEXT OUTSIDE THE PROVIDED MACROS
%

%
%    - All parameters are checked at the verification or submission.
%    - Some parameters are also checked during the pdfLaTeX
%      compilation.  If this is not the case, this is indicated by the
%      phrase
%      "This parameter is NOT checked at the pdfLaTeX compilation."
%

\documentclass{esoform}

% The list of LaTeX definitions of commonly used astronomical symbols
% is already included in the style file common2e.sty (see Table 1 in
% the Users' Manual).  If you have your own macros or definitions,
% please insert them here, between the \documentclass{esoform}
% and the \begin{document} commands.
%
%     PLEASE USE NEITHER YOUR OWN MACROS NOR ANY TEX/LATEX MACROS  
%       IN THE \Title, \Abstract, \PI, \CoI, and \Target MACROS.
%
% WARNING: IT IS THE RESPONSIBILITY OF THE APPLICANTS TO STAY WITHIN THE
% CURRENT BOX LIMITS AND ELIMINATE POTENTIAL OVERFILL/OVERWRITE PROBLEMS 

\begin{document}

%%%%%%%%%%%%%%%%%%%%%%%%%%%%%%%%%%%%%%%%%%%%%%%%%%%%%%%%%%%%%%%%%%%%%%%%
%%%%% CONTENTS OF THE FIRST PAGE %%%%%%%%%%%%%%%%%%%%%%%%%%%%%%%%%%%%%%%
%%%%%%%%%%%%%%%%%%%%%%%%%%%%%%%%%%%%%%%%%%%%%%%%%%%%%%%%%%%%%%%%%%%%%%%%
%
%---- BOX 1 ------------------------------------------------------------
%
% You should use this template for period 96A applications ONLY.
%
% DO NOT EDIT THE MACRO BELOW. 

\Cycle{96A}

% Type below, within the curly braces {}, the title of your observing
% programme (up to two lines).
% This parameter is NOT checked at the pdfLaTeX compilation.
%
% DO NOT USE ANY TEX/LATEX MACROS IN THE TITLE

\Title{Hubble Frontier Field Supernova Spectroscopy}  

% Type below the numeric code corresponding to the subcategory of your
% programme.

\SubCategoryCode{A4}   

% Please specify the type of programme you are submitting. 
% Valid values: NORMAL, GTO, TOO, CALIBRATION, MONITORING
% If you specify TOO, you will also need to fill a ToO page below.
% If you specify CALIBRATION, then the SubCategory Code MUST be set to L0

% If your programme requires more than 100 hours the Large Programme
% template (templatelarge.tex) must be used.


\ProgrammeType{TOO}

% For GTO proposals only: uncomment the following and fill out the GTO
% programme code (as communicated to the respective GTO coordinator).

%\GTOcontract{INS-consortium}		

% For TOO proposals only: uncomment the following if you apply for
% Rapid Response Mode observations.
 
%\ObservationInRRM{}

% Uncomment the following macro if this proposal is applying for time
% under the VLT-XMM agreement (only available for odd periods).

%\ObservationWithXMM{}

%---- BOX 2 ------------------------------------------------------------
%
% Type below a concise abstract of your proposal (up to 9 lines).
% This parameter is NOT checked at the pdfLaTeX compilation.
%
% DO NOT USE ANY TEX/LATEX MACROS IN THE ABSTRACT

\Abstract{
The HST Frontier Fields (HFF) program present an extraordinary opportunity for the detection of lensed, high redshift supernovae (SNe) out to $z\sim 3$. We propose to capitalize on this unique asset by searching the HFF data and triggering ToO follow-up for SNe of interest. Over the first 7 Frontier Fields observing windows we have found 17 SNe, including the multiply imaged SN Refsdal, and we expect this number to increase. The number of detected high-$z$ SNe Ia is small, but carry great leverage for testing progenitor model through the delay time distribution while the lensed SNe Ia offer a unique chance to validate cluster mass models by directly measuring the lensing magnification. We will also be able to extend core-collapse SN rate measurements for the first time beyond $z\sim 1$. This follow-up program provides the spectroscopic information necessary to unlock the science potential of these SNe.
}

%---- BOX 3 ------------------------------------------------------------
%
% Description of the observing run(s) necessary for the completion of
% your programme.  The macro takes ten parameters: run ID, period,
% instrument, time requested, month preference, moon requirement,
% seeing requirement, transparency requirement, observing mode and 
% run type.
%
% 1. RUN ID
% Valid values: A, B, ..., Z
% Please note that only one run per intrument is allowed for APEX
%
% 2. PERIOD
% Valid values: 96
% Exceptions:
% Monitoring Programmes: These programmes can span up to four periods.
%
% VLT-XMM proposals: These are only accepted in odd periods and are 
% also valid for the next period.
%
% This parameter is NOT checked at the pdfLaTeX compilation.
%
% 3. INSTRUMENT
% Valid values: AMBER ARTEMIS EFOSC2 FLAMES FLASH FORS2 HARPS HAWKI KMOS LABOCA MUSE NACO OMEGACAM PIONIER SEPIA SHFI SINFONI SOFI SOFOSC SPHERE Special3.6 SpecialAPEX SpecialNTT UVES VIMOS VIRCAM VISIR XSHOOTER
% 
% Only Chilean and GTO Programmes are accepted on OMEGACAM.
% No normal programmes on OMEGACAM will be accepted.
% Please note that only a subset of these instruments will be accepted
% for Monitoring Programmes. Please see the Call for Proposals and the
% ESOFORM User Manual for more details.
%
% 4. TIME REQUESTED
% In hours for Service Mode, in nights for Visitor Mode.
% In either case the time can be rounded up to  1 decimal place. 
% This parameter is NOT checked at the pdfLaTeX compilation.
% 
% 5. MONTH PREFERENCE
% Valid values: oct, nov, dec, jan, feb, mar, any
%
% 6. MOON REQUIREMENT
% Valid values: d, g, n
%
% 7. SEEING REQUIREMENT
% Valid values: 0.4, 0.6, 0.8, 1.0, 1.2, 1.4, n
%
% 8. TRANSPARENCY REQUIREMENT
% Valid values: CLR, PHO, THN
%
% 9. OBSERVING MODE
% Valid values: v, s
%
% 10. RUN TYPE
% Valid values: TOO 
% For all Normal & Calibration Programmes this field should be blank.
% For TOO & GTO Programmes, users can specify TOO runs.
% If the field is left blank a default normal, non-TOO run is assumed.
% If a TOO run is specified please make sure that you fill in the TOO page.




\ObservingRun{A}{96}{XSHOOTER}{6.4h}{any}{n}{1.0}{THN}{s}{TOO}



% Proprietary time requested.
% Valid values: % 0, 1, 2, 6, 12

\ProprietaryTime{12}

%---- BOX 4 ------------------------------------------------------------
%
% Indicate below the telescope(s) and number of nights/hours already
% awarded to this programme, if any.
% This macro is optional and can be commented out.
% It is also NOT checked at the pdfLaTeX compilation.

\AwardedNights{XSHOOTER}{20.2h awarded in P92--P93.}

% Indicate below the telescope(s) and number of nights/hours still
% necessary, in the future, to complete this programme, if any.
% This macro is optional and can be commented out.
% It is also NOT checked at the pdfLaTeX compilation.

\FutureNights{XSHOOTER}{$\sim$ 10 h in total (P97)}

%---- BOX 5 ------------------------------------------------------------
%
% Take advantage of this box to provide any special remark  (up to three
% lines). In case of coordinated observations with XMM, please specify
% both the ESO period and the preferred month for the XMM
% observations here.
% This macro is optional and can be commented out.
% It is also NOT checked at the pdfLaTeX compilation.

\SpecialRemarks{This proposal is part of a larger collaborative effort, and involves a dedicated 60-orbit HST program (PI Rodney) to improve the characterisation of candidate SNe discovered in the Hubble Frontier Fields plus additional orbits dedicated to the follow-up of SN Refsdal in MACSJ1149, all piggybacking on 840 orbits of HST time for the Frontier Fields. Some of the spectroscopy could 
possibly be obtained in 'normal' service mode (see Box 8b).}
  
%---- BOX 6 ------------------------------------------------------------
% Please provide the ESO User Portal username for the Principal
% Investigator (PI) in the \PI field.
%
% For the Co-I's (CoI) please fill in the following details:
% First and middle initials, family name, the institute code
% corresponding to their affiliation. 
% The corresponding affiliation should be entered for EACH
% Co-I separately in order to ensure the correct details of 
% all Co-I's are stored in the ESO database.
% You can find all institute codes listed according to country
% on the following webpage:
% http://www.eso.org/sci/observing/phase1/countryselect.html
%
% For example, if the Co-I's full name is David Alan William Jones,
% his affiliation is the Observatoire de Paris, Site de Paris, 
% you should write:
% \CoI{D.A.W.}{Jones}{1588}
% Further examples are shown below.
% DO NOT USE ANY TEX/LATEX MACROS HERE
%

\PI{JHJORTH}
% Replace with PI's ESO User Portal username.

\CoI{S.}{Rodney}{1698}
\CoI{J.}{Selsing}{1227}
\CoI{L.}{Christensen}{1227}

% Please note: 
% Due to the way in which the proposal receiver system parses
% the CoI macro, the number of pairs of curly brackets '{}'
% in this macro MUST be strictly equal to 3, i.e., the
% number of parameters of the macro. Accordingly, curly
% brackets should not be used within the parameters (e.g.,
% to protect LaTeX signs).
%
% For instance:
% \CoI{L.}{Ma\c con}{1098}
% \CoI{R.}{Men\'endez}{1098}
%
% are valid, while
%
% \CoI{L.}{Ma{\c}con}{1098}
% \CoI{R.}{Men{\'}endez}{1098}
%
% are not. Unfortunately the receiver does not give an
% explicit error message when such invalid forms are
% used in the CoI macro, but the processing of the proposal
% keeps hanging indefinitely.


%%%%%%%%%%%%%%%%%%%%%%%%%%%%%%%%%%%%%%%%%%%%%%%%%%%%%%%%%%%%%%%%%%%%%%%%
%%%%% THE TWO PAGES OF THE SCIENTIFIC DESCRIPTION AND FIGURES %%%%%%%%%%
%%%%%%%%%%%%%%%%%&&&%%%%%%%%%%%%%%%%%%%%%%%%%%%%%%%%%%%%%%%%%%%%%%%%%%%%
%
%---- BOX 7 ------------------------------------------------------------
%
%               THIS DESCRIPTION IS RESTRICTED TO TWO PAGES 
%
%   THE RELATIVE LENGTHS OF EACH OF THE SECTIONS ARE VARIABLE,
%   BUT THEIR SUM (INCLUDING FIGURES & REFS.) IS RESTRICTED TO TWO PAGES
%
% All macros in this box are NOT checked at the pdfLaTeX compilation.

\ScientificRationale{

Over the past decade, the deep HST Treasury surveys (GOODS, CANDELS, CLASH) 
have all enabled ``piggyback'' Type Ia SN searches (hereafter, the 
HST-SN surveys), which have collectively accumulated scores of SN detections 
that reach to uniquely high redshifts (Riess et al.\ 2007, ApJ, 659, 98; 
Dahl\'en et al.\ 2008, ApJ, 681, 462; Suzuki et al.\ 2012, ApJ, 746, 85). The ongoing HFF
program now provides a powerful new tool for the discovery of 
particularly high-$z$ SNe. What sets this survey apart from the previous 
HST-SN surveys is the unique depth of each visit, reaching 
$m_{lim,3\sigma}({\rm F160W}) \approx 27.9$ (AB), $\sim 1$ mag deeper than 
CANDELS/CLASH per epoch. Gravitational lensing in the prime fields can also 
magnify fluxes by about a factor of 2, making it possible to detect distant 
background events.  {\it The Hubble Frontier Field survey thus provides the 
first opportunity to discover SNe at $2<z<3$, building up a small but 
important ``New Frontier'' sample.}

\medskip\noindent

There is a
long-standing and uncomfortable ambiguity about the nature of SNIa
progenitor systems, and the mechanisms by which they reach the point
of explosion (e.g., Maoz \& Mannucci 2012,
PASA, 29, 447). Most models assume that the primary progenitor star is 
a white dwarf (WD), but they differ on the identity of the companion 
(mass donating) star, the time scale for mass-accretion (or merger), and 
ultimately the time required to go from progenitor formation to explosion 
(or the delay time). There are two principal families of models: the 
``double degenerate'' models, in which orbital decay causes two WDs in 
a binary system to merge; and the ``single degenerate'' models, wherein 
the WD accretes mass gradually from a main sequence or evolved giant 
companion star. It is unclear what scenario(s) are ultimately responsible 
for the creation of Ia SNe, but resolving this question has tremendous
astrophysical implications touching on heavy element enrichment over cosmic 
history, the origins of galactic winds and galaxy formation, and the 
robustness of SNIa as standard candles over cosmic time.

%There is an uncomfortable ambiguity about the nature of SN Ia progenitor 
%systems (e.g., Maoz \& Mannucci 2012, PASA, 29, 447). Models differ on
%the identity of the companion star, the time scale for mass-accretion
%(or merger), and 
%%ultimately 
%the time from progenitor formation to
%explosion (the ``delay time'').  Resolving the progenitor problem has
%tremendous astrophysical implications touching on heavy element
%enrichment over cosmic history, the origins of galactic winds and
%galaxy formation, and the robustness of SNe Ia as standard candles.

\medskip\noindent
Volumetric SN Ia rates provide important constraints on the progenitor systems.  
Each scenario predicts different delay-time distributions, which can 
be constrained by measuring the SN Ia rate with $z$. Our HST-SN surveys
now reach $z \approx 2$ with large uncertainties 
(Graur et al., 2014, ApJ, 783, 28; Rodney et al., 2014, ApJ, 148, 13
see Figure~1). It is here that the HFFs will have the greatest impact, adding about 5 
new SNe Ia, and increasing the $z>1.5$ sample size by $\sim$ 50\%.

\medskip\noindent

%\medskip\noindent

Observations of SNe lensed by clusters of galaxies allows a measure of the true lensing magnifications and thereby a critical test for the existing mass models. The mass models are based on assumptions about the distribution of dark matter and confronting these will significantly improve our understanding og the nature of dark matter. A strongly lensed SN Ia at $z = 1.31$, have recently been observed in exactly our program (Rodney et al.\ 2015, in prep.) and with a magnification $\mu = 2.0$, we have found that all existing mass models for this cluster significantly overestimates the magnification.  This also has important implications for the study of lensed $z > 8$ galaxies discovered (Zheng, W., et al. 2012, Nature, 489, 406) in these clusters in terms of the inferred properties which are affected by the magnification assumed. 






%Gravitational lensing in the prime fields can also magnify fluxes, making it easier to detect distant background events as is the case for SN Tomas (Rodney et al.\ 2015, in prep.) in which a magnification of $\mu = 2.0$ is found for a SN Ia at $z = 1.31$.


%in that our best estimates are that we predict $3\pm3$ suitable SNe in P94.
%Nevertheless, based on detailed Monte Carlo simulations and comparisons to 
%HST-SN survey yields, we expect that the pyramid of ground + GLASS + HST-SN HFF 
%should yield us about 50 SNe of all types over the 3-year survey, with about 
%5 SNe Ia at $z > 1.5$, competitive with the entirety of 
%the CLASH program and equal to about half of the CANDELS program.


%
%Volumetric SNIa rates can provide important constraints on the dominant 
%(or sole) progenitor mechanism. Each scenario predicts delay time 
%distributions for a population of progenitors born from a burst of star 
%formation, $\Phi(t)$, that can be convolved with the star formation rate 
%density history, $\dot{\rho}_{\star}(z)$, to get a model of volumetric 
%SNIa rates, $R_{\rm Ia}(z)$, which can then be compared to measurements.
%As shown in Figure~2, there are many viable models that can provide very 
%good fits to the measured rates at $z<1$. However, these models begin to 
%diverge at $z>1$, so that high-$z$ regime has the greatest leverage for 
%discriminating a dominant model. Unfortunately this redshift range is 
%currently starved for statistics, due to the low yield of Ia events seen 
%in these high-$z$ surveys (Dahl\'en et al.\ 2008; Graur et al.\ 2011,
%MNRAS, 417, 916).
}

\ImmediateObjective{Our primary science goal is to provide a real measurement
of SN Ia rates at $z > 1.0$, yielding improved constraints on SN Ia progenitor
models.  A secondary goal is to build a sample of lensed
SNe Ia to provide independent validation tests of cluster lensing
models. Furthermore, this sample will
extend the measurement of CC SN rates to $z>1$,
find SNe Ia for cosmological analyses, 
and possibly discover rare events such as multiply lensed SNe. The latter have recently been achieved when our collaboration discovered \textit{the first ever multiply imaged transient} dubbed SN Refsdal (Kelly, P. L., et al. 2015, Science, 347, 1123).

\medskip\noindent

As the HFF HST data are obtained -- 840 HST orbits in total at the end of the program, we are searching for SNe at all redshifts. We process the HFF data through a pipeline that we have developed 
and battle-tested for SN discovery in the HST-SN surveys. When a SN has been detected the HST ToO program is triggered to obtain a photo-$z$ for the host galaxy and a rough estimate of the redshift from the SN colours. From the photometry we assess whether we can determine a spectroscopic redshift and in some cases classify the event from an X-shooter spectrum.  We have already been awarded 60-orbit HST ToO follow-up (PID 13386,13790, PI Rodney) to observe SN lightcurves as well as 5-10 orbits of IR imaging monitoring MACS1149 for a re-appearance of the multiply imaged, SN Refsdal.

\medskip\noindent

In its comments to our previous proposals, the OPC noted the uncertain number of events and the thereby potential lack of quantitative constraints. We agree that the rates in any given period are uncertain, but given the proven unique leverage of these events, we believe that the importance of spectroscopic follow-up have been shown. In P92 and P93 we triggered the VLT on four events. Based on our detailed simulations we expect $3\pm3$ SNe to be detectable with VLT during P96, but given past experience we will likely only trigger on 2 systems, which is what we apply for here. We will observe either the SNe or the host galaxies, or both. In terms of observational efficiency, the host is much more reliable, as we typically have good emission lines for our high-z host targets, while the SN spectrum requires template matching to broad absorption features. In our previous X-shooter programs, most redshifts were indeed obtained through host galaxy spectroscopy and we expect the same to be the case for this X-shooter/HFF program. However, a SN spectrum is valuable for classification, and is critical in cases where the SN is "hostless" or the host association is ambiguous, about 15\% of the time. This is highlighted for SN Preston (Figure 2) where the capability of doing ToO spectroscopy proved invaluable for typing and redshift determination.  A prime example of this is shown in Figure 2.

\medskip\noindent
{\bf SN Ia Rates:}
The HST-SN sample in the highest redshift bin currently comprises
only $\sim$11 objects: 3 from GOODS and 8 from the nearly completed
CANDELS+CLASH survey. The expected yield of $\sim$5 new SNe Ia at
$z>1.5$ from the HFFs therefore represents an increase in
the sample size of nearly 50\%.  This could reduce the overall
measurement error (statistical and systematic) from 55\% to $\sim$30\%.  
With this improved precision we could potentially reject any models 
dominated by very prompt explosions (see Figure 1).  

\medskip\noindent
{\bf Testing Lens Models:} 
We expect to find several magnified SNe Ia behind the HFF
clusters, and by measuring their light curve shapes with our follow-up orbits 
we will determine their absolute luminosities to $\pm0.2$ mags, providing a 
direct measurement of the lensing magnification (Patel et al., 2014, ApJ, 786, 9; Nordin, J., et al. 2014, MNRAS, 440, 2742). The precision of these SN Ia magnification measures will be on par with the magnification predictions available from existing mass models. Agreement between the SNe and the mass models will provide an important validation of the models. Any disagreement can provide interesting insights into the local clumpiness of the dark matter halo. For a proof of concept, this exercise has been done for a strongly lensed SN Ia at $z = 1.31$ (Rodney et al.\ 2015, in prep.) with an inferred magnification $\mu = 2.0 \pm 0.19$, significantly different from predictions of current mass models.

\medskip\noindent
{\bf Star Formation From CC SNe:}
CC SN volumetric rates have a much closer connection to the star-formation rate
density than do SNe Ia, because CC SNe have essentially no delay
from formation to explosion. CC SN rates, therefore, allow a means for 
measuring the star formation rate history independent from galaxy UV 
luminosities (Dahl\'en et al.\ 2012, ApJ, 757, 70) and will allow a 
direct comparison of 
CC SN rates and star formation rates (Horiuchi et al.\ 2011, ApJ, 738, 154;
Madau \& Dickinson, ARAA, 2014. 52:415–86). 
Approximately half of the HFF discoveries will be CC SNe, allowing us to 
extend CC SN rates measurements to $z>1$, probing closer to the peak of the 
cosmic star formation rate history.  
\vspace{-0.4cm}
 } 

%
%---- THE SECOND PAGE OF THE SCIENCE CASE CAN INCLUDE FIGURES ----------
%
% Up to ONE page of figures can be added to your proposal.  
% The text and figures of the scientific description must not
% exceed TWO pages in total. 
% If you use color figures, do make sure that they are still readable
% if printed in black and white. Figures must be in PDF or JPEG format.
% Each figure has a size limit of 1MB.
% MakePicture and MakeCaption are optional macros and can be commented out.

\MakePicture{ratesDTDall.jpg}{angle=0,width=13cm, height = 6cm}
\MakeCaption{\vspace{-1.0cm} Figure~1. 
{\it SN Ia rates constrain progenitor models.}
%{\bf Left}: Our X-shooter observations of the [O III] 5007 line in an
%HST-SN host galaxy, defining the current frontier for Ia SNe 
%($z=1.91$) (Frederiksen 2013, PhD thesis; Jones et al. 2013, ApJ, 768, 166).
%{\bf Middle}: Histograms showing the predicted distribution of SN 
%detections for all SN types over the full 3-year HFF survey. Using Monte 
%Carlo simulations of the HFF survey, and informed by our experience with 
%CANDELS and CLASH, we predict that the total yield of SN targets accessible 
%to VLT will be about 5--6 per period for 6 periods. 
%{\bf Right}: 
Measured volumetric SN Ia rates as a function of redshift, comparing observed 
SN rates against two progenitor models from Rodney et al.\ (2014).  
Large purple diamonds show the CLASH rates from Graur et al.\ (2014) and 
large orange squares show the CANDELS rates from Rodney et al.\ (2014). Two curves show the SN Ia rates predicted by models that allow for a variable fraction of 
SN Ia explosions to occur promptly, within 500 Myr after their formation. The ground-based observations (black points) prefer the model, which has $\sim$60\% of all SNe Ia exploding within 500 Myr. The green dashed line is the best-fit model when using the CANDELS+CLASH data alone, for which the prompt fraction is $\sim$20\%. The HFF SN sample will increase the sample size at $z > 1.5$ by about 50\%, and will extend the measurement for the first time to $z \approx 3$, providing unique leverage for 
discriminating between these models.
\vspace{-0.4cm}
}

\MakePicture{SN_Preston.png}{angle=0,width=16cm, height = 6cm}
\MakeCaption{\vspace{-0.8cm} Figure~2. 
{Left: \it X-shooter spectrum of SN Preston.}
%{\bf Left}: Our X-shooter observations of the [O III] 5007 line in an
%HST-SN host galaxy, defining the current frontier for Ia SNe 
%($z=1.91$) (Frederiksen 2013, PhD thesis; Jones et al. 2013, ApJ, 768, 166).
%{\bf Middle}: Histograms showing the predicted distribution of SN 
%detections for all SN types over the full 3-year HFF survey. Using Monte 
%Carlo simulations of the HFF survey, and informed by our experience with 
%CANDELS and CLASH, we predict that the total yield of SN targets accessible 
%to VLT will be about 5--6 per period for 6 periods. 
%{\bf Right}: 
In a 3.2h X-shooter spectrum we detect clear SN-features after a host template spectrum has been subtracted. Right: The photometric redshift of the host is $z \sim 0.83$ which agrees with the \textit{SALT2 SN Ia light curve fit}. This is also in rough agreement with the spectroscopically determined redshift of a SN Ia at $ z \sim 0.9$ obtained from template matching with  SNID (Blondin \& Tonry, 2007, ApJ, 666, 1024). This highlights the need for ToO observations, giving an independent measure of type and redshift. To our knowledge this is the highest redshift Ia typed from a ground-based spectrum.
}

%%%%%%%%%%%%%%%%%%%%%%%%%%%%%%%%%%%%%%%%%%%%%%%%%%%%%%%%%%%%%%%%%%%%%%
%%%%% THE PAGE OF TECHNICAL JUSTIFICATIONS %%%%%%%%%%%%%%%%%%%%%%%%%%%%%
%%%%%%%%%%%%%%%%%%%%%%%%%%%%%%%%%%%%%%%%%%%%%%%%%%%%%%%%%%%%%%%%%%%%%%%%
%
%---- BOX 8 ------------------------------------------------------------
%
% Provide below a careful justification of the requested lunar phase
% and of the requested number of nights or hours.  
% All macros in this box are NOT checked at the pdfLaTeX compilation.

\WhyLunarPhase{The driver for this proposal is time-critical (ToO)
NIR observations and we therefore do not wish to put any restrictions on
the lunar phase.}  

\WhyNights{For the SNe, we assume peak magnitudes of $M_B=-19.34$ and a 
standard cosmology ($\Omega_m=0.3,\Omega_\Lambda =0.7,h=0.7$). At 
$z=1$ and $z=1.5$ the Si II $\lambda$6150 feature is redshifted to the 
J and H bands, respectively. At $z=1$ J=23.25 (H=23.5) and at $z=1.5$ 
H=23.7. Assuming a restframe width of 150 \AA\ of Si II $\lambda$6150, this 
can be detected at 5--10$\sigma$ at $z=1$ (depending on the profile shape 
and depth) and half that at $z=1.5$ according to the X-shooter 
ETC v.5.0.1 in $8\times 1200$ s, corresponding to 3.2h execution time. 
%We will trigger on H $< 23.7$ sources, although in
%most cases we will require H $< 23.5$, in some cases (e.g., at $z=0.6$)
%significantly brighter.  
Hence, for the SNe we adopt exposure times of 3.2h.
For host galaxies which are assumed to be emission line sources our 
experience is that redshifts can sometimes be obtained in a 
$4\times 1200$ s ABBA sequence (1.6h execution time, as used in
previous periods), while at higher redshifts, usually twice that
may be required (e.g., our determination of $z=1.91$ for a SN Ia host). 
Given that we are probing new territory and higher redshifts, we expect 
that 3.2h will be required in most cases. See Figure 2 for proof of concept.
}

\TelescopeJustification{X-shooter is by far the most efficient instrument in 
the world to perform ground-based spectroscopy of high-redshift SNe.}

\ModeJustification{ToO observations require service mode. 
%In its comments to our P92 proposal, the OPC remarked: 
%"For a future resubmission, the proposers may consider whether the proposal 
%(or some subset of runs) could be `normal' rather than `ToO', since the 
%latter is currently defined by ESO as cases where the target is known less 
%than 1 week in advance of the observations being required, whereas for some 
%of the proposed observations a 1 week turnaround would be acceptable.''. 
%It is true that a 1 week turnaround is sufficient in some cases, but we 
%need the valuable ability to occasionally do a quick ToO that informs our 
%HST follow-up choices. If we can rely on turnarounds approaching 7 days 
%for `normal' scheduling this would be an interesting possibility in about 
%half the cases and we would be happy to adopt such an approach. However, we 
%have no experience with such fast turnaround normal service observations and 
%do not know the procedures for dealing with this on a routine basis.
In its comments to our P94 proposal, the OPC remarked: "The primary goal of the observations is redshifts (rather than typing), and will make use of the SN themselves, or their hosts. In principle, this could all be done targeting the hosts after". It is true that for some targets the redshift could be observed in "normal" mode and in some cases a 1 week turnaround would be sufficient, but we need the valuable ability to occasionally do a quick ToO that informs our HST follow-up choices. If we can rely on turnarounds approaching 7 days for 'normal' scheduling this would be an interesting possibility in about half the cases and we would be happy to adopt such an approach. However, we have no experience with such fast turnaround normal service observations and do not know the procedures for dealing with this on a routine basis. As can be seen in Figure 2, we have targets where the ToO capability proves invaluable for typing and redshift determination. 

}


% Please specify the type of calibrations needed.
% In case of special calibration the second parameter is used to enter 
% specific details.
% Valid values: standard, special
%\Calibrations{special}{Adopt a special calibration}
\Calibrations{standard}{}


%%%%%%%%%%%%%%%%%%%%%%%%%%%%%%%%%%%%%%%%%%%%%%%%%%%%%%%%%%%%%%%%%%%%%%%
%% PAGE OF BOXES 9-10  %%%%%%%%%%%%%%%%%%%%%%%%%%%%%%%%%%%%%%%%%%%%%%%%
%%%%%%%%%%%%%%%%%%%%%%%%%%%%%%%%%%%%%%%%%%%%%%%%%%%%%%%%%%%%%%%%%%%%%%%
%
%---- BOX 9 -- Use of ESO Facilities --------------------------------
%
% This macro is optional and can be commented out.
% It is also NOT checked at the pdfLaTeX compilation.
% LastObservationRemark: Report on the use of the ESO facilities during
%  the last 2 years (4 observing periods). Describe the status of the
%  data obtained and the scientific output generated.

\LastObservationRemark{
{\bf 092.A-0533,093.A-0667 (PI Hjorth)}:
So far, 20.2h of X-shooter TOO time has been awarded to this effort; 14.2h
have been used on four triggers, yielding a redshift of 0.41 for a cluster SN Ia, a $z=1.32$ SN Ia in MACS1423, a SN Ia at $z\sim0.9$ in the parallel field of Abell 2744 (see Figure 2) and SN Spock is pending observation. 


\smallskip\noindent
{\bf 090.A-0726 (PI Hjorth)}:
%{\bf 087.A-0419, 088.A-0708, 089.A-0739, 090.A-0726 (PI Hjorth)}:
High-redshift CANDELS/CLASH SNe and host galaxies. 
%X-shooter GTO programme for high-redshift supernovae.
%A total of 29.7 h have been used. The first high-redshift Ia SN are reported
%in Rodney et al., Frederiksen et al. ab, Jones et al..
%The programme has resulted in several redshifts for high-redshift supernovae,
%including the record of $z=2.37$. The full X-shooter sample is presented in
This program has led to several redshift determinations of high-redshift
SNe, both Ia and CC, and in some cases have allowed a 
detailed study of the properties of the SN host galaxies:
%Rodney et al.: ``A Type Ia supernova at redshift 1.55 in Hubble Space Telescope 
%infrared observations from CANDELS'', ApJ, 746, 5, 2012,
%Frederiksen et al.: ``The dwarf starburst host galaxy of a Type Ia
%supernova at $z=1.55$ from CANDELS'', ApJ, 760, 125, 2012, 
%Frederiksen et al.: ``Spectroscopic Identification of a Redshift 1.55 
%Supernova Host Galaxy from the Subaru Deep Field Supernova Survey'', 
%A\&A, in press. The redshift of the most distant Ia is based on
%HST grism and X-shooter spectroscopy:
%Jones et al.: ``The Discovery of the Most Distant Known Type Ia 
%Supernova at Redshift 1.914'', ApJ, 768, 166, 2013. The inferred SN rates
Rodney et al., ApJ, 746, 5, 2012;
Frederiksen et al., ApJ, 760, 125, 2012;
Frederiksen et al., A\&A, 2014, 563, A140.
The redshift of the most distant SN Ia is based on
HST grism and X-shooter spectroscopy:
Jones et al., ApJ, 768, 166, 2013. The inferred SN rates
are presented in Graur et al., ApJ, 783, 28, 2014 and Rodney et al., 2014, ApJ, 148, 13.
%A detailed study of high-redshift SN host galaxies observed by 
%X-shooter is the subject of Frederiksen's PhD thesis (2013) and resulting
%papers in preparation.

\smallskip\noindent
{\bf 090.D-0719 (PI Hjorth)}: Supernova dust extinction. Results so far
include Gall et al., ATel 4454, 2012;
Margutti et al., ApJ, 780, 21, 2014;
Gall et al., Nature 511, 326 - 329, 2014
%A panchromatic view of the restless SN 2009ip reveals
%the explosive ejection of a massive star envelope
%Rapid formation of large dust grains in the luminous supernova SN 2010jl
}

%
%---- BOX 9a -- ESO Archive ------------------------------------------
%
% Are the data requested in this proposal in the ESO Archive
% (http://archive.eso.org)? If yes, explain the need for new data.
% This macro is NOT checked at the pdfLaTeX compilation.

\RequestedDataRemark{This is a ToO proposal. The data requested
are not in the archive.}

%
%---- BOX 9b -- ESO GTO/Public Survey Programme Duplications---------
%
% If any of the targets/regions in ongoing GTO Programmes and/or
% Public Surveys are being duplicated here, please explain why.
% This macro is optional and can be commented out.
% It is also NOT checked at the pdfLaTeX compilation.

\RequestedDuplicateRemark{
 There is no duplication of targets/regions covered 
 by ongoing GTO and/or Public Survey programmes. 
 }

%
%---- BOX 10 ------ Applicant(s) publications ---------------------
%
% Applicant's publications related to the subject of this proposal
% during the past two years.  Use the simplified abbreviations for
% references as in A&A.  Separate each reference with the following
% usual LaTex command: \smallskip\\
%   
%   Name1 A., Name2 B., 2001, ApJ, 518, 567: Title of article1
%   \smallskip\\
%   Name3 A., Name4 B., 2002, A\&A, 388, 17: Title of article2
%   \smallskip\\
%   Name5 A. et al., 2002, AJ, 118, 1567: Title of article3
%
% This macro is NOT checked at the pdfLaTeX compilation.

\Publications{
Frederiksen T. F. et al., 2012, ApJ, 760, 125:
The dwarf starburst host galaxy of a Type Ia supernova at $z=1.55$ from CANDELS
  \smallskip\\
Frederiksen T. F. et al., 2014, A\&A, 563, A140:
Spectroscopic identification of a redshift 1.55 supernova host galaxy from 
the Subaru Deep Field supernova survey
  \smallskip\\
Gall C. et al., 2014, Nature 511, 326 - 329:
Rapid formation of large dust grains in the luminous supernova SN 2010jl
  \smallskip\\
Graur O. et al., 2014, ApJ, 783, 28:
Type-Ia supernova rates to redshift 2.4 from CLASH: 
The cluster lensing and supernova survey with Hubble
  \smallskip\\
Hjorth, J. et al., 2012, ApJ, 756, 187:
The Optically Unbiased Gamma-ray burst host (TOUGH) survey. I. Survey design 
and catalogs
  \smallskip\\
Li, Hjorth \& Wojtak, 2014, ApJl,  accepted (arXiv:1409.3567):
Cosmological Parameters From Supernovae Associated With Gamma-ray Bursts
%Hjorth, J. et al., 2012, Chapter 9 in ``Gamma-Ray Bursts", Cambridge 
%Astrophysics Series 51, eds.\ C. Kouveliotou, R. A. M. J. Wijers and 
%S. Woosley, Cambridge University Press (Cambridge), pp.\ 169--190: 
%The gamma-ray burst -- supernova connection
  \smallskip\\
Hjorth, J., 2013, Phil.\ Trans.\ R. Soc.\ A, 20120275:
The supernova--gamma-ray burst--jet connection
  \smallskip\\
Jones D. et al., 2013, ApJ, 768, 166:
The discovery of the most distant known Type Ia supernova at redshift 1.914
  \smallskip\\
Margutti, R. et al., 2014, ApJ, 780, 21:
A panchromatic view of the restless SN 2009ip reveals
the explosive ejection of a massive star envelope
  \smallskip\\
Rodney S. et al., 2012, ApJ, 746, 5:
A Type Ia supernova at redshift 1.55 in Hubble Space Telescope infrared 
observations from CANDELS
  \smallskip\\
Rodney et al., 2014, ApJ, 148, 13:
Type Ia supernova rate measurements to redshift 2.5 from CANDELS: 
Searching for prompt explosions in the early universe
}

%%%%%%%%%%%%%%%%%%%%%%%%%%%%%%%%%%%%%%%%%%%%%%%%%%%%%%%%%%%%%%%%%%%%%%%%
%%%%% THE PAGE OF THE TARGET/FIELD LIST %%%%%%%%%%%%%%%%%%%%%%%%%%%%%%%%
%%%%%%%%%%%%%%%%%%%%%%%%%%%%%%%%%%%%%%%%%%%%%%%%%%%%%%%%%%%%%%%%%%%%%%%%
%
%---- BOX 11 -----------------------------------------------------------
%
% Complete list of targets/fields requested.  The macro takes nine
% parameters: run ID, target field/name, RA, Dec, time on target, magnitude, 
% diameter, additional information, reference star.
%
% 1. RUN ID
% Valid values: run IDs specified in BOX 3
%
% 2. TARGET FIELD/NAME
%
% 3. RA (J2000)
% Format: hh mm ss.f, or hh mm.f, or hh.f
% Use 00 00 00 for unknown coordinates
% This parameter is NOT checked at the pdfLaTeX compilation.
% 
% 4. Dec (J2000)
% Format: dd mm ss, or dd mm.f, or dd.f
% Use 00 00 00 for unknown coordinates
% This parameter is NOT checked at the pdfLaTeX compilation.
%
% 5. TIME ON TARGET
% Format: hours (overheads and calibration included)
% This parameter is NOT checked at the pdfLaTeX compilation.
%
% 6. MAGNITUDE
% This parameter is NOT checked at the pdfLaTeX compilation.
%
% 7. ANGULAR DIAMETER
% This parameter is NOT checked at the pdfLaTeX compilation.
%
% 8. ADDITIONAL INFORMATION
% Any relevant additional information may be inserted here.
% For APEX and CRIRES runs, the requested PWV upper limit MUST
% be specified for each target using this field.
% For APEX runs, the acceptable LST range MUST also be specified here.
% This parameter is NOT checked at the pdfLaTeX compilation.
%
% 9. REFERENCE STAR ID
% See Users' Manual.
% This parameter is NOT checked at the pdfLaTeX compilation.
%
% Long lists of targets will continue on the last page of the
% proposal.
%
%                       ** VERY IMPORTANT ** 
% The scheduling of your programme will take into account ALL targets
% given in this list. INCLUDE ONLY TARGETS REQUESTED FOR P96 
% (except for runs outside P96, e.g. for VLT-XMM and Monitoring proposals).
%
% DO NOT USE ANY TEX/LATEX MACROS FOR THE TARGETS


\Target{A}{SN1}{00 00 00.0}{+00 00 00}{3.2}{J=23}{1"}{}{}
\Target{A}{SN2}{00 00 00.0}{+00 00 00}{3.2}{J=23}{1"}{}{}


%                      ***************** 
%                      ** PWV limits **
% For CRIRES and all APEX instruments users must specify the PWV upper
% limits for each target. For example:
%\Target{}{Alpha CMa}{06 45 08.9}{-16 42 58}{1}{-1.4}{6 mas}{PWV=1.0mm, Sirius}{}
%\Target{}{HD 104237}{12 00 05.6}{-78 11 33}{1}{}{}{PWV<0.7mm;LST=9h00-15h00}{}
%
%                      *****************

% Use TargetNotes to include any comments that apply to several or all
% of your targets.
% This macro is NOT checked at the pdfLaTeX compilation.

\TargetNotes{
The targets will primarily be in the clusters of galaxies 
Abell S1063 (RA(J2000) = 22:48:44.4, Dec(J2000) = $-44$:31:48.5), 
Abell 370 (RA(J2000) = 02:39:52.4, Dec(J2000) = $-01:$34:36.5),
or their flanking fields a few arcmin away.  The monitoring of SN Refdal in MACSJ1149 (RA(J2000) = 11:49:36.3, Dec(J2000) = $+22$:23:58.1) could potentially lead to more event, or
other clusters may
be observed from the ground and will also provide rare events.
Targets will be selected as being strong SN candidates based on 
colours and light curve properties in the HST imaging. Targets may be 
either the SNe themselves or their host galaxies (or both). We 
reserve the right to observe each target for shorter or longer than 
then default 3.2h specified above, depending on brightness and results of 
first inspections of the data (e.g., if the observations have to be split 
into two visits). Each OB will be a 1.6h execution time OB, consisting 
of $4\times1200$ s (ABBA) in the NIR arm, with an AFC template in the middle.
}

%%%%%%%%%%%%%%%%%%%%%%%%%%%%%%%%%%%%%%%%%%%%%%%%%%%%%%%%%%%%%%%%%%%%%%%%
%%%%% TWO PAGES OF SCHEDULING REQUIREMENTS %%%%%%%%%%%%%%%%%%%%%%%%%%%%%
%%%%%%%%%%%%%%%%%%%%%%%%%%%%%%%%%%%%%%%%%%%%%%%%%%%%%%%%%%%%%%%%%%%%%%%%
%
%---- BOX 12 -----------------------------------------------------------
%

% Uncomment the following line if the proposal involves time-critical
% observations, or observations to be performed at specific time
% intervals. Please leave these brackets blank. Details of time
% constraints can be entered in Special Remarks and using the
% other flags in Box 13.
%
%
%\HasTimingConstraints{}

%
% The timing constraint macros listed below 
% are optional and can be commented out:
% \HasTimingConstraints, \RunSplitting, \Link and \TimeCritical
% They are also NOT checked at the pdfLaTeX compilation.


% 1. RUN SPLITTING, FOR A GIVEN ESO TELESCOPE (Visitor Mode only)
%
% 1st argument: run ID
% Valid values: run IDs specified in BOX 3
%
% 2nd argument: run splitting requested for sub-runs
% This parameter is NOT checked at the pdfLaTeX compilation.

%\RunSplitting{B}{1,10s,1}
%\RunSplitting{C}{2,10s,2,20w,2,15s,4H2}


% 2. LINK FOR COORDINATED OBSERVATIONS BETWEEN DIFFERENT RUNS.
%\Link{B}{after}{A}{10}
%\Link{C}{after}{B}{}
%\Link{E}{simultaneous}{F}{}

% 3. UNSUITABLE PERIOD(S) OF TIME
%
% 1st argument: run ID
% Valid values: run IDs specified in BOX 3
%
% 2nd argument: Chilean start date for the unsuitable time
% Format: dd-mmm-yyyy
% This parameter is NOT checked at the pdfLaTeX compilation.
%
% 3rd argument: Chilean end date for the unsuitable time
% Format: dd-mmm-yyyy
% This parameter is NOT checked at the pdfLaTeX compilation.

%\UnsuitableTimes{A}{15-jan-16}{18-jan-16}{Insert explanation of unsuitable time here.}
%\UnsuitableTimes{B}{15-jan-16}{18-jan-16}{Insert explanation of unsuitable time here.}
%\UnsuitableTimes{C}{20-jan-16}{23-jan-16}{Insert explanation of unsuitable time here.}


%
%---- BOX 12 contd.. -- Scheduling Requirements 
%

% SPECIFIC DATE(S) FOR TIME-CRITICAL OBSERVATIONS
% Please note: The dates must correspond to the LOCAL CHILEAN observing dates.
%
% The 2nd and 3rd parameters are NOT checked at the pdfLaTeX compilation.
% 1st argument: run ID
% Valid values: run IDs specified in BOX 3
%
% 2nd argument: Chilean start date for the critical period.
% Format: dd-mmmm-yyyy 
%
% 3rd argument: Chilean end date for the critical period.
% Format: dd-mmmm-yyyy

%\TimeCritical{A}{12-nov-15}{14-nov-15}{Insert reason for time-critical observations.}
%\TimeCritical{D}{1-nov-15}{2-nov-15}{Insert reason for time-critical observations.}
%\TimeCritical{D}{17-nov-15}{18-nov-15}{Insert reason for time-critical observations.}
%\TimeCritical{D}{23-nov-15}{24-nov-15}{Insert reason for time-critical observations.}



%%%%%%%%%%%%%%%%%%%%%%%%%%%%%%%%%%%%%%%%%%%%%%%%%%%%%%%%%%%%%%%%%%%%%%%%
%
%---- BOX 14 -----------------------------------------------------------
%
% INSTRUMENT CONFIGURATIONS:
%
% Uncomment only the lines related to instrument configuration(s)
% needed for the acquisition of your planned observations. 
%
% 1st argument: run ID
% Valid values: run IDs specified in BOX 3
%
% 2nd argument: instrument
% This parameter is NOT checked at the pdfLaTeX compilation.
%
% 3rd argument: mode
% This parameter is NOT checked at the pdfLaTeX compilation.
%
% 4th argument: additional information
% This parameter is NOT checked at the pdfLaTeX compilation.
%
% All parameters are mandatory and cannot be empty. Do NOT specify
% Instrument Configurations for alternative runs.

% Examples (to be commented or deleted)

\INSconfig{A}{XSHOOTER}{300-2500nm}{SLT}
\INSconfig{A}{XSHOOTER}{SLT}{1.0, 0.9, 0.9JH}
\INSconfig{A}{XSHOOTER}{SLT}{100k-1x1,100k-1x1,NDR}
%
% Real list of instrument configurations

%%%%%%%%%%%%%%%%%%%%%%%%%%%%%%%%%%%%%%%%%%%%%%%%%%%%%%%%%%%%%%%%%%%%%%%%%
% Paranal
%
%-----------------------------------------------------------------------
%---- NAOS/CONICA at the VLT-UT1 (ANTU)  -------------------------------
%-----------------------------------------------------------------------
%
%\INSconfig{}{NACO}{PRE-IMG}{provide list of filters HERE}
%
% Specify the NGS name, distance from target and magnitude  
%(Vmag preferred, otherwise Rmag) in the target list,
% and uncomment the following line
%\INSconfig{}{NACO}{NGS}{-}
%
%\INSconfig{}{NACO}{Special Cal}{Select if you have special calibrations}
%\INSconfig{}{NACO}{Pupil Track}{Select if you need pupil tracking mode}
%\INSconfig{}{NACO}{Cube}{Select if you need cube mode}
%
%\INSconfig{}{NACO}{SAM VIS-WFS}{Provide list of masks and filters HERE}
%\INSconfig{}{NACO}{SAM IR-WFS}{Provide list of masks and filters HERE}
%\INSconfig{}{NACO}{SAMPol VIS-WFS}{Provide list of masks and filters HERE}
%\INSconfig{}{NACO}{SAMPol IR-WFS}{Provide list of masks and filters HERE}
%
%\INSconfig{}{NACO}{IMG 54 mas/px IR-WFS}{provide list of filters HERE}
%\INSconfig{}{NACO}{IMG 27 mas/px IR-WFS}{provide list of filters HERE}
%\INSconfig{}{NACO}{IMG 13 mas/px IR-WFS}{provide list of filters HERE}
%\INSconfig{}{NACO}{IMG 54 mas/px VIS-WFS}{provide list of filters HERE}
%\INSconfig{}{NACO}{IMG 27 mas/px VIS-WFS}{provide list of filters HERE}
%\INSconfig{}{NACO}{IMG 13 mas/px VIS-WFS}{provide list of filters HERE}
%
%\INSconfig{}{NACO}{CORONA AGPM VIS-WFS}{provide list of filters (L',NB-3.74,NB-4.05) HERE}
%\INSconfig{}{NACO}{CORONA AGPM IR-WFS}{provide list of filters (L',NB-3.74,NB-4.05) HERE}
%
%\INSconfig{}{NACO}{POL 54 mas/px IR-WFS}{provide list of filters HERE}
%\INSconfig{}{NACO}{POL 27 mas/px IR-WFS}{provide list of filters HERE}
%\INSconfig{}{NACO}{POL 13 mas/px IR-WFS}{provide list of filters HERE}
%\INSconfig{}{NACO}{POL 54 mas/px VIS-WFS}{provide list of filters HERE}
%\INSconfig{}{NACO}{POL 27 mas/px VIS-WFS}{provide list of filters HERE}
%\INSconfig{}{NACO}{POL 13 mas/px VIS-WFS}{provide list of filters HERE}
%
%\INSconfig{}{NACO}{SPEC IR-WFS}{provide the list of spectroscopic modes HERE}
%\INSconfig{}{NACO}{SPEC VIS-WFS}{provide the list of spectroscopic modes HERE}
% % 
%

%-----------------------------------------------------------------------
%---- FORS2 at the VLT-UT1 (ANTU) --------------------------------------
%-----------------------------------------------------------------------
%If you require the E2V (Blue) detector uncomment the following line
%\INSconfig{}{FORS2}{Detector}{E2V}
%
%If you require the MIT (RED) detector uncomment the following line
%\INSconfig{}{FORS2}{Detector}{MIT}
%
% If you require the High-Resolution  collimator uncomment the following line
%\INSconfig{}{FORS2}{collimator}{HR}
%
% Uncomment the line(s) corresponding to the imaging mode(s) you require and
% provide the list of filters needed  for your observations:
%
%\INSconfig{}{FORS2}{PRE-IMG}{ESO filters: provide list HERE}
%\INSconfig{}{FORS2}{IMG}{ESO filters: provide list HERE}
%\INSconfig{}{FORS2}{IMG}{User's own filters (to be described in text)}
%\INSconfig{}{FORS2}{IPOL}{ESO filters: provide list HERE}
%\INSconfig{}{FORS2}{IPOL}{User's own filters (to be described in text)}
%\INSconfig{}{FORS2}{HIT-MS}{Provide list of grisms HERE}
%
%
% Uncomment the line(s) corresponding to the spectroscopic mode(s) you require and
% provide the list of grism+filter combination needed  for your observations:
%
%\INSconfig{}{FORS2}{LSS}{Provide list of grism+filter combinations HERE}
%\INSconfig{}{FORS2}{MOS}{Provide list of grism+filter combinations HERE}
%\INSconfig{}{FORS2}{PMOS}{Provide list of grism+filter combinations HERE}
%\INSconfig{}{FORS2}{MXU}{Provide list of grism+filter combinations HERE}
%\INSconfig{}{FORS2}{HITI}{ESO filters: provide list HERE}
%\INSconfig{}{FORS2}{HIT-OS}{Provide list of grisms HERE}
%
% Uncomment the following line for Rapid Response Mode observations
%
%\INSconfig{}{FORS2}{RRM}{yes}
%
%-----------------------------------------------------------------------
%---- KMOS at the VLT-UT1 (ANTU) ---------------------------------------
%-----------------------------------------------------------------------
%
%\INSconfig{}{KMOS}{IFU}{provide list of settings (IZ, YJ, H, K, HK) here} 
%
%-----------------------------------------------------------------------
%---- FLAMES at the VLT-UT2 (KUEYEN) -----------------------------------
%-----------------------------------------------------------------------
%\INSconfig{}{FLAMES}{UVES}{Specify the UVES setup below}
%\INSconfig{}{FLAMES}{GIRAFFE-MEDUSA}{Specify the GIRAFFE setup below}
%\INSconfig{}{FLAMES}{GIRAFFE-IFU}{Specify the GIRAFFE setup below}
%\INSconfig{}{FLAMES}{GIRAFFE-ARGUS}{Specify the GIRAFFE setup below}
%\INSconfig{}{FLAMES}{Combined: UVES + GIRAFFE-MEDUSA}{Specify the UVES and
%GIRAFFE setups below}
%\INSconfig{}{FLAMES}{Combined: UVES + GIRAFFE-IFU}{Specify the UVES and
%GIRAFFE setups below}
%\INSconfig{}{FLAMES}{Combined: UVES + GIRAFFE-ARGUS}{Specify the UVES and
%GIRAFFe setups below}
%
%
% If you have selected UVES, either standalone or in combined mode,
% please specify the UVES standard setup(s) to be used
%\INSconfig{}{FLAMES}{UVES}{standard setup Red 520}
%\INSconfig{}{FLAMES}{UVES}{standard setup Red 580}
%\INSconfig{}{FLAMES}{UVES}{standard setup Red 580 + simultaneous calibration}
%\INSconfig{}{FLAMES}{UVES}{standard setup Red 860}
%
%\INSconfig{}{FLAMES}{GIRAFFE}{fast readout mode 625kHz VM only}
%
% If you have selected GIRAFFE, either standalone or in combined mode
% please specify the GIRAFFE standard setups(s) to be used
%\INSconfig{}{FLAMES}{GIRAFFE}{standard setup HR01 379.0}
%\INSconfig{}{FLAMES}{GIRAFFE}{standard setup HR02 395.8}
%\INSconfig{}{FLAMES}{GIRAFFE}{standard setup HR03 412.4}
%\INSconfig{}{FLAMES}{GIRAFFE}{standard setup HR04 429.7}
%\INSconfig{}{FLAMES}{GIRAFFE}{standard setup HR05 447.1 A}
%\INSconfig{}{FLAMES}{GIRAFFE}{standard setup HR05 447.1 B}
%\INSconfig{}{FLAMES}{GIRAFFE}{standard setup HR06 465.6}
%\INSconfig{}{FLAMES}{GIRAFFE}{standard setup HR07 484.5 A}
%\INSconfig{}{FLAMES}{GIRAFFE}{standard setup HR07 484.5 B}
%\INSconfig{}{FLAMES}{GIRAFFE}{standard setup HR08 504.8}
%\INSconfig{}{FLAMES}{GIRAFFE}{standard setup HR09 525.8 A}
%\INSconfig{}{FLAMES}{GIRAFFE}{standard setup HR09 525.8 B}
%\INSconfig{}{FLAMES}{GIRAFFE}{standard setup HR10 548.8}
%\INSconfig{}{FLAMES}{GIRAFFE}{standard setup HR11 572.8}
%\INSconfig{}{FLAMES}{GIRAFFE}{standard setup HR12 599.3}
%\INSconfig{}{FLAMES}{GIRAFFE}{standard setup HR13 627.3}
%\INSconfig{}{FLAMES}{GIRAFFE}{standard setup HR14 651.5 A}
%\INSconfig{}{FLAMES}{GIRAFFE}{standard setup HR14 651.5 B}
%\INSconfig{}{FLAMES}{GIRAFFE}{standard setup HR15 665.0}
%\INSconfig{}{FLAMES}{GIRAFFE}{standard setup HR15 679.7}
%\INSconfig{}{FLAMES}{GIRAFFE}{standard setup HR16 710.5}
%\INSconfig{}{FLAMES}{GIRAFFE}{standard setup HR17 737.0 A}
%\INSconfig{}{FLAMES}{GIRAFFE}{standard setup HR17 737.0 B}
%\INSconfig{}{FLAMES}{GIRAFFE}{standard setup HR18 769.1}
%\INSconfig{}{FLAMES}{GIRAFFE}{standard setup HR19 805.3 A}
%\INSconfig{}{FLAMES}{GIRAFFE}{standard setup HR19 805.3 B}
%\INSconfig{}{FLAMES}{GIRAFFE}{standard setup HR20 836.6 A}
%\INSconfig{}{FLAMES}{GIRAFFE}{standard setup HR20 836.6 B}
%\INSconfig{}{FLAMES}{GIRAFFE}{standard setup HR21 875.7}
%\INSconfig{}{FLAMES}{GIRAFFE}{standard setup HR22 920.5 A}
%\INSconfig{}{FLAMES}{GIRAFFE}{standard setup HR22 920.5 B}
%\INSconfig{}{FLAMES}{GIRAFFE}{standard setup LR01 385.7}
%\INSconfig{}{FLAMES}{GIRAFFE}{standard setup LR02 427.2}
%\INSconfig{}{FLAMES}{GIRAFFE}{standard setup LR03 479.7}
%\INSconfig{}{FLAMES}{GIRAFFE}{standard setup LR04 543.1}
%\INSconfig{}{FLAMES}{GIRAFFE}{standard setup LR05 614.2}
%\INSconfig{}{FLAMES}{GIRAFFE}{standard setup LR06 682.2}
%\INSconfig{}{FLAMES}{GIRAFFE}{standard setup LR07 773.4}
%\INSconfig{}{FLAMES}{GIRAFFE}{standard setup LR08 881.7}
%
%\INSconfig{}{FLAMES}{GIRAFFE}{fast readout mode 625kHz VM only}
%
%-----------------------------------------------------------------------
%---- X-SHOOTER at the VLT-UT2 (KUEYEN)
%-----------------------------------------------------------------------
%
%\INSconfig{}{XSHOOTER}{300-2500nm}{SLT}
%\INSconfig{}{XSHOOTER}{300-2500nm}{IFU}
%
% Slits (SLT only):
%
%UVB arm, available slits in arcsec: 0.5, 0.8, 1.0, 1.3, 1.6, 5.0
%VIS arm, available slits in arcsec: 0.4, 0.7, 0.9, 1.2, 1.5, 5.0 
%NIR arm, available slits in arcsec: 0.4, 0.6, 0.6JH, 0.9, 0.9JH, 1.2, 5.0
%  The 0.6JH and 0.9JH include a stray light K-band blocking filter
%  that allow sky limited studies in J and H bands.
%
%The slits for IFU  are fixed and do not need to be mentioned here.
%
% Replace SLIT-UVB, SLIT-VIS, SLIT-NIR with the choice of the slits:
%\INSconfig{}{XSHOOTER}{SLT}{SLIT-UVB,SLIT-VIS,SLIT-NIR}
%
% Detector readout mode:
%
% UVB and VIS arms: available readout modes and binning:
% 100k-1x1, 100k-1x2, 100k-2x2, 400k-1x1, 400k-1x2, 400k-2x2
% The NIR readout mode is fixed  to NDR.
%
%\INSconfig{}{XSHOOTER}{IFU}{readout UVB,readout VIS,readout NIR}
%\INSconfig{}{XSHOOTER}{SLT}{readout UVB,readout VIS,readout NIR}
%
% Imaging mode 
% replace 'list of filters' by the actual filters you wish to use among:
% U, B, V, R, I, Uprime, Gprime, Rprime, Iprime, Zprime
% Please note that the imaging mode can only be used in combination with slit or IFU observations
%\INSconfig{}{XSHOOTER}{IMG}{list of filters}
%
%\INSconfig{}{XSHOOTER}{RRM}{yes}
%
%-----------------------------------------------------------------------
%---- UVES at the VLT-UT2 (KUEYEN) -------------------------------------
%-----------------------------------------------------------------------
%
%\INSconfig{}{UVES}{BLUE}{Standard setting: 346}
%\INSconfig{}{UVES}{BLUE}{Standard setting: 437}
%\INSconfig{}{UVES}{BLUE}{Non-std setting: provide central wavelength  HERE}
%
%\INSconfig{}{UVES}{RED}{Standard setting: 520}
%\INSconfig{}{UVES}{RED}{Standard setting: 580}
%\INSconfig{}{UVES}{RED}{Standard setting: 600}
%\INSconfig{}{UVES}{RED}{Iodine cell standard setting: 600}
%\INSconfig{}{UVES}{RED}{Standard setting: 860}
%\INSconfig{}{UVES}{RED}{Non-std setting: provide central wavelength HERE}
%
%\INSconfig{}{UVES}{DIC-1}{Standard setting: 346+580}
%\INSconfig{}{UVES}{DIC-1}{Standard setting: 390+564}
%\INSconfig{}{UVES}{DIC-1}{Standard setting: 346+564}
%\INSconfig{}{UVES}{DIC-1}{Standard setting: 390+580}
%\INSconfig{}{UVES}{DIC-1}{Non-std setting: provide central wavelength HERE}
%
%\INSconfig{}{UVES}{DIC-2}{Standard setting: 437+860}
%\INSconfig{}{UVES}{DIC-2}{Standard setting: 346+860}
%\INSconfig{}{UVES}{DIC-2}{Standard setting: 390+860}
%
%\INSconfig{}{UVES}{DIC-2}{Standard setting: 437+760}
%\INSconfig{}{UVES}{DIC-2}{Standard setting: 346+760}
%\INSconfig{}{UVES}{DIC-2}{Standard setting: 390+760}
%\INSconfig{}{UVES}{DIC-2}{Non-std setting: provide central wavelength HERE}
%
%\INSconfig{}{UVES}{Field Derotation}{yes}
%\INSconfig{}{UVES}{Image slicer-1}{yes}
%\INSconfig{}{UVES}{Image slicer-2}{yes}
%\INSconfig{}{UVES}{Image slicer-3}{yes}
%\INSconfig{}{UVES}{Iodine cell}{yes}
%\INSconfig{}{UVES}{Longslit Filters}{Provide list of filters HERE}
%
%\INSconfig{}{UVES}{RRM}{yes}
%
%
%-----------------------------------------------------------------------
%---- SPHERE at the VLT-UT3 (MELIPAL) -----------------------------------
%-----------------------------------------------------------------------
%
%
% Pupil or field tracking?
% Mode choices: IRDIS-CI, IRDIS-DBI, 
%               IRDIFS, IRDIFS-EXT, 
%               ZIMPOL-I
%               (Not relevant for IRDIS-DPI, IRDIS-LSS, ZIMPOL-P1 or ZIMPOL-P2)
%--------------------
% IRDIFS: 
% Coronagraph combination choices:
%   IRDIFS:     None, N-ALC-YJH-S, N-ALC-YJH-L, N-CLC-SW-L, N-4Q-YJH
%   IRDIFS-EXT: None, N-ALC-YJH-S, N-ALC-YJH-L, N-ALC-Ks
% Filter choices for IRDIS in IRDIFS mode
%   IRDIFS:     DB-H23, DB-ND23, DB-H34
%   IRDIFS-EXT: DB-K12
%---------------------
% IRDIS imaging (alone):
% Coronagraph combination choices for IRDIS imaging modes (see UM for details)
%   IRDIS-CI, IRDIS-DPI:  
%              None, N-ALC-Y, N-ALC-YJ-S, N-ALC-YJ-L, N-ALC-YJH-S, 
%                    N-ALC-YJH-L, N-ALC-Ks, N-4Q-YJH, N-4Q-Ks
%   IRDIS-DBI: None, N-ALC-Y, N-ALC-YJ-S, N-ALC-YJ-L, N-ALC-YJH-S, 
%                    N-ALC-YJH-L, N-ALC-Ks, N-4Q-YJH, N-4Q-Ks
% Filter choices:
%   IRDIS-CI, IRDIS-DPI: 
%              BB-Y, BB-J, BB-H, BB-Ks, NB-Hel, NB-CntJ, NB-CntH,
%              NB-CntK1, NB-BrG, NB-CntK2, NB-PaB, NB-FeII, NB-H2, NB-CO
%   IRDIS-DBI: DB-Y23, DB-J23, DB-H23, DB-NDH23,  DB-H34, DB-K12 
%---------------------
% IRDIS spectroscopy:
% Coronagraphic slit/grism combinations for IRDIS-LSS:
%   IRDIS-LSS: N-S-LR-WL, N-S-MR-WL
%---------------------
% ZIMPOL imaging: 
% Coronagraph choices:
%   ZIMPOL-I: None, V-CLC-M-WF, V-CLC-M-NF, V-CLC-L-WF, V-CLC-XL-WF
% Filter choices:
%   ZIMPOL-I: RI, R-PRIM, I-PRIM, V, V-S, V-L, N-R, 730-NB, N-I, I-L,
%             KI,  TiO-717, CH4-727, Cnt748, Cnt820, HeI, OI-630,
%             CntHa, B-Ha, N-Ha, Ha-NB
%--------------------
% ZIMPOL polarimetry:
% Coronagraph choices:
%    ZIMPOL-P1: None, V-CLC-S-WF, V-CLC-M-WF, V-CLC-L-WF, V-CLC-XL-WF, V-CLC-MT-WF
%    ZIMPOL-P2: None, V-CLC-S-WF, V-CLC-M-WF, V-CLC-L-WF, V-CLC-XL-WF, V-CLC-MT-WF
% Filter choices:
%    ZIMPOL-P1: RI, R-PRIM, I-PRIM, V, N-R, N-I, KI, TiO-717, 
%               CH4-727, Cnt748, Cnt820, CntHa, N-Ha, B-Ha     
%    ZIMPOL-P2: RI, R-PRIM, I-PRIM, V, N-R, N-I, KI, TiO-717, 
%               CH4-727, Cnt748, Cnt820, CntHa, N-Ha, B-Ha 
% Readout mode choice for ZIMPOL
%    ZIMPOL-P1: FastPol, SlowPol
%    ZIMPOL-P2: FastPol, SlowPol
%-------------------
%
% One entry per mode. Repeat the entry for each mode.
%
%\INSconfig{}{SPHERE}{Pupil}{mode}
%\INSconfig{}{SPHERE}{Field}{mode}
%
% One entry per combination. Repeat the entry for each combination.
%
%\INSconfig{}{SPHERE}{IRDIFS}{Coronagraph/filter combination for IRDIFS}
%\INSconfig{}{SPHERE}{IRDIFS-EXT}{Coronagraph/filter combination for IRDIFS-EXT}
%
%\INSconfig{}{SPHERE}{IRDIS-CI}{Coronagraph/filter combination for IRDIS-CI}
%\INSconfig{}{SPHERE}{IRDIS-DBI}{Coronagraph/filter combination for IRDIS-DBI}
%\INSconfig{}{SPHERE}{IRDIS-DPI}{Coronagraph/filter combination for IRDIS-DPI}
%\INSconfig{}{SPHERE}{IRDIS-LSS}{Coronagraphic slit/grism combination for IRDIS-LSS}
%
%\INSconfig{}{SPHERE}{ZIMPOL-I}{Coronagraph/filter combination for ZIMPOL-I}
%
%\INSconfig{}{SPHERE}{ZIMPOL-P1}{Coronagraph/filter/readout mode for ZIMPOL-P1}
%\INSconfig{}{SPHERE}{ZIMPOL-P2}{Coronagraph/filter/readout mode for ZIMPOL-P2}
%
%-----------------------------------------------------------------------
%---- VISIR at the VLT-UT3 (MELIPAL) -----------------------------------
%-----------------------------------------------------------------------
%
%\INSconfig{}{VISIR}{IMG N-band 45 mas/px}{Provide list of filters HERE}
%\INSconfig{}{VISIR}{IMG Q-band 45 mas/px}{Provide list of filters HERE}
%
%% AGPM: pending successful commissioning early March 2015.
%     list of filters to be provided on the VISIR web page by March 15
%\INSconfig{}{VISIR}{CORONA AGPM 45 mas/px}{List of filters}
%
%\INSconfig{}{VISIR}{SPEC N-band LR}{-}
%\INSconfig{}{VISIR}{SPEC N-band HR Longslit}{Provide central wavelengt(s) (8.02,12.81) HERE}
%\INSconfig{}{VISIR}{SPEC Q-band HR Longslit}{Provide central wavelength(s) (17.03) HERE}
%\INSconfig{}{VISIR}{SPEC N-band HRCrossdispersed}{Provide central wavelength(s) (7.7-13.3)}
%\INSconfig{}{VISIR}{SPEC Q-band HRCrossdispersed}{Provide central wavelength(s) (16.0-24.0) HERE}
%
%-----------------------------------------------------------------------
%---- VIMOS at the VLT-UT3 (MELIPAL) -----------------------------------
%-----------------------------------------------------------------------
%
%\INSconfig{}{VIMOS}{PRE-IMG}{ESO filters: enter the list of filters}
%\INSconfig{}{VIMOS}{IMG}{ESO filters: enter the list of filters}
%\INSconfig{}{VIMOS}{IFU 0.67"/fibre}{LR-Red}
%\INSconfig{}{VIMOS}{IFU 0.67"/fibre}{LR-Blue}
%\INSconfig{}{VIMOS}{IFU 0.67"/fibre}{MR}
%\INSconfig{}{VIMOS}{IFU 0.67"/fibre}{HR-Red}
%\INSconfig{}{VIMOS}{IFU 0.67"/fibre}{HR-Orange}
%\INSconfig{}{VIMOS}{IFU 0.67"/fibre}{HR-Blue}
%
%\INSconfig{}{VIMOS}{IFU 0.33"/fibre}{LR-Red}
%\INSconfig{}{VIMOS}{IFU 0.33"/fibre}{LR-Blue}
%\INSconfig{}{VIMOS}{IFU 0.33"/fibre}{MR}
%\INSconfig{}{VIMOS}{IFU 0.33"/fibre}{HR-Red}
%\INSconfig{}{VIMOS}{IFU 0.33"/fibre}{HR-Orange}
%\INSconfig{}{VIMOS}{IFU 0.33"/fibre}{HR-Blue}
%
%\INSconfig{}{VIMOS}{MOS-grisms}{LR-Red}
%\INSconfig{}{VIMOS}{MOS-grisms}{LR-Blue}
%\INSconfig{}{VIMOS}{MOS-grisms}{MR}
%\INSconfig{}{VIMOS}{MOS-grisms}{HR-Red}
%\INSconfig{}{VIMOS}{MOS-grisms}{HR-Orange}
%\INSconfig{}{VIMOS}{MOS-grisms}{HR-Blue}
%
%\INSconfig{}{VIMOS}{MOS-slits-targets}{0.6" < slit width < 1.4", targets:stellar}
%\INSconfig{}{VIMOS}{MOS-slits-targets}{0.6" < slit width < 1.4", targets:extended}
%\INSconfig{}{VIMOS}{MOS-slits-targets}{slit width > 1.4", targets:stellar}
%\INSconfig{}{VIMOS}{MOS-slits-targets}{slit width > 1.4", targets:extended}
%\INSconfig{}{VIMOS}{MOS-masks}{Enter here number of mask sets (1 set = 4 quadrants)}
%
%
%%-----------------------------------------------------------------------
%---- HAWKI at the VLT-UT4 (YEPUN) -----------------------------------
%-----------------------------------------------------------------------
%
%\INSconfig{}{HAWKI}{PRE-IMG}{provide list of filters (Y,J,H,Ks,CH4,BrG,H2,NB1190,NB1060,NB2090) HERE}
%\INSconfig{}{HAWKI}{IMG}{provide list of filters (Y,J,H,Ks,CH4,BrG,H2,NB1190,NB1060,NB2090) HERE}
%\INSconfig{}{HAWKI}{BURST}{Provide list of filters  (Y,J,H,Ks,CH4,BrG,H2,NB1190,NB1060,NB2090) HERE}
%\INSconfig{}{HAWKI}{FASTJITT}{Provide list of filters  (Y,J,H,Ks,CH4,BrG,H2,NB1190,NB1060,NB2090) HERE}
%\INSconfig{}{HAWKI}{RRM}{yes}
%
%-----------------------------------------------------------------------
%---- SINFONI at the VLT-UT4 (YEPUN) -----------------------------------
%-----------------------------------------------------------------------
%

%\INSconfig{}{SINFONI}{PRE-IMG}{provide list of setting(s) (J,H,K,H+K)}
%
%\INSconfig{}{SINFONI}{IFS 250mas/pix no-AO}{provide list of setting(s) (J,H,K,H+K) HERE}
%\INSconfig{}{SINFONI}{IFS 100mas/pix no-AO}{provide list of setting(s) (J,H,K,H+K) HERE}
%
% If you plan to use a NGS, please specify the NGS name and magnitude (Rmag preferred,
% otherwise Vmag) in target list.
%\INSconfig{}{SINFONI}{IFS 250mas/pix NGS}{provide list of setting(s) (J,H,K,H+K) HERE}
%\INSconfig{}{SINFONI}{IFS 100mas/pix NGS}{provide list of setting(s) (J,H,K,H+K) HERE}
%\INSconfig{}{SINFONI}{IFS 25mas/pix NGS}{provide list of setting(s) (J,H,K,H+K) HERE}
%
% If you plan to use the LGS, please specify the TTS name and magnitude (Rmag preferred,
% otherwise Vmag) in target list.
%\INSconfig{}{SINFONI}{IFS 250mas/pix LGS}{provide list of setting(s) (J,H,K,H+K) HERE}
%\INSconfig{}{SINFONI}{IFS 100mas/pix LGS}{provide list of setting(s) (J,H,K,H+K) HERE}
%\INSconfig{}{SINFONI}{IFS 25mas/pix LGS}{provide list of setting(s) (J,H,K,H+K) HERE}
%
% If you plan to use the LGS without a TTS (seeing enhancer mode) then
% please leave the TTS name blank in the target list.
%\INSconfig{}{SINFONI}{IFS 250mas/pix LGS-noTTS}{provide list of setting(s) (J,H,K,H+K) HERE}
%\INSconfig{}{SINFONI}{IFS 100mas/pix LGS-noTTS}{provide list of setting(s) (J,H,K,H+K) HERE}
%\INSconfig{}{SINFONI}{IFS 25mas/pix LGS-noTTS}{provide list of setting(s) (J,H,K,H+K) HERE}
%
% Select if you have special calibrations
%\INSconfig{}{SINFONI}{Special Cal}{-}
%
% Select if you need pupil tracking mode
%\INSconfig{}{SINFONI}{Pupil Track}{-}
%
% Select for RRM
%\INSconfig{}{SINFONI}{RRM}{yes}
%
%-----------------------------------------------------------------------
%---- MUSE at the VLT-UT4 (YEPUN) -----------------------------------
%-----------------------------------------------------------------------
%
%\INSconfig{}{MUSE}{WFM-NOAO-N}{-}
%\INSconfig{}{MUSE}{WFM-NOAO-E}{-}
%
% Uncomment the following line for Rapid Response Mode observations
%\INSconfig{}{MUSE}{RRM}{yes}
%
%%%%%%%%%%%%%%%%%%%%%%%%%%%%%%%%%%%%%%%%%%%%%%%%%%%%%%%%%%%%%%%%%%%%%%%%
%-----------------------------------------------------------------------
%---- AMBER ------------------------------------------------------------
%-----------------------------------------------------------------------
%
%\INSconfig{}{AMBER}{LR-HK}{2.2}
%\INSconfig{}{AMBER}{LR-HK-F}{2.2}
%
%\INSconfig{}{AMBER}{MR-K}{2.1}
%\INSconfig{}{AMBER}{MR-K-F}{2.1}
%
%\INSconfig{}{AMBER}{MR-H}{1.65}   
%\INSconfig{}{AMBER}{MR-H-F}{1.65} 
%
%\INSconfig{}{AMBER}{MR-K}{2.3}
%\INSconfig{}{AMBER}{MR-K-F}{2.3}
%
%\INSconfig{}{AMBER}{HR-K}{Central wavelength selected from the list:
% 1.97929,2.01786,2.05643,2.09500,2.13357,2.17214,2.21071,2.24929,2.28786,2.32643,
% 2.36500,2.40357,2.44214,2.48071}
%\INSconfig{}{AMBER}{HR-K-F}{Central wavelength selected from the list:
% 1.97929,2.01786,2.05643,2.09500,2.13357,2.17214,2.21071,2.24929,2.28786,2.32643,
% 2.36500,2.40357,2.44214,2.48071}
% 
%where *-F means with FINITO
%

%-----------------------------------------------------------------------
%---- PIONIER ----------------------------------------------------------
%-----------------------------------------------------------------------
%
%
%\INSconfig{}{PIONIER}{GRISM}{1.65}
%\INSconfig{}{PIONIER}{FREE}{1.65}
%
%%%%%%%%%%%%%%%%%%%%%%%%%%%%%%%%%%%%%%%%%%%%%%%%%%%%%%%%%%%%%%%%%%%%%%%%
%
%-----------------------------------------------------------------------
%---- VIRCAM at VISTA --------------------------------------------------
%-----------------------------------------------------------------------
%
%\INSconfig{}{VIRCAM}{IMG}{provide list of filters here}
%
%-----------------------------------------------------------------------
%---- OMEGACAM at VST --------------------------------------------------
% This instrument is only available for GTO, Chilean and filler programmes.
%-----------------------------------------------------------------------
%
%\INSconfig{}{OMEGACAM}{IMG}{provide list of filters here}
%
%%%%%%%%%%%%%%%%%%%%%%%%%%%%%%%%%%%%%%%%%%%%%%%%%%%%%%%%%%%%%%%%%%%%%%%%
% La Silla
%-----------------------------------------------------------------------
%---- EFOSC2 (or SOFOSC) at the NTT ------------------------------------
%-----------------------------------------------------------------------
%
%\INSconfig{}{EFOSC2}{PRE-IMG}{EFOSC2 filters: provide list here}
%\INSconfig{}{EFOSC2}{Imaging-filters}{EFOSC2 filters:  provide list here}
%\INSconfig{}{EFOSC2}{Imaging-filters}{ESO non EFOSC filters: provide ESOfilt No}
%\INSconfig{}{EFOSC2}{Imaging-filters}{User's own filters (to be described in text)}
%\INSconfig{}{EFOSC2}{Spectro-long-slit}{Grism\#1:320-1090}
%\INSconfig{}{EFOSC2}{Spectro-long-slit}{Grism\#2:510-1100}
%\INSconfig{}{EFOSC2}{Spectro-long-slit}{Grism\#3:305-610}
%\INSconfig{}{EFOSC2}{Spectro-long-slit}{Grism\#4:409-752}
%\INSconfig{}{EFOSC2}{Spectro-long-slit}{Grism\#5:520-935}
%\INSconfig{}{EFOSC2}{Spectro-long-slit}{Grism\#6:386-807}
%\INSconfig{}{EFOSC2}{Spectro-long-slit}{Grism\#7:327-524}
%\INSconfig{}{EFOSC2}{Spectro-long-slit}{Grism\#8:432-636}
%\INSconfig{}{EFOSC2}{Spectro-long-slit}{Grism\#11:338-752}
%\INSconfig{}{EFOSC2}{Spectro-long-slit}{Grism\#13:369-932}
%\INSconfig{}{EFOSC2}{Spectro-long-slit}{Grism\#14:310-509}
%\INSconfig{}{EFOSC2}{Spectro-long-slit}{Grism\#16:602-1032}
%\INSconfig{}{EFOSC2}{Spectro-long-slit}{Grism\#17:689-876}
%\INSconfig{}{EFOSC2}{Spectro-long-slit}{Grism\#18:470-677}
%\INSconfig{}{EFOSC2}{Spectro-long-slit}{Grism\#19:440-510}
%\INSconfig{}{EFOSC2}{Spectro-long-slit}{Grism\#20:605:715}
%\INSconfig{}{EFOSC2}{Spectro-long-slit}{Aperture: 0.5'', ... ,10.0''}
%
%\INSconfig{}{EFOSC2}{Spectro-long-slit}{Aperture: Shiftable}
%\INSconfig{}{EFOSC2}{Spectro-MOS}{PunchHead=0.95''}
%\INSconfig{}{EFOSC2}{Spectro-MOS}{PunchHead=1.12''}
%\INSconfig{}{EFOSC2}{Spectro-MOS}{PunchHead=1.45''}
%\INSconfig{}{EFOSC2}{Polarimetry}{$\lambda / 2$ retarder plate}
%\INSconfig{}{EFOSC2}{Polarimetry}{$\lambda / 4$ retarder plate}
%\INSconfig{}{EFOSC2}{Coronograph}{yes}
%
%
%-----------------------------------------------------------------------
%---- SOFI (or SOFOSC) at the NTT --------------------------------------------------
%-----------------------------------------------------------------------
%
%\INSconfig{}{SOFI}{PRE-IMG-LargeField}{Provide list of filters HERE}
%\INSconfig{}{SOFI}{Imaging-LargeField}{Provide list of filters HERE}
%\INSconfig{}{SOFI}{Burst}{Provide list of filters HERE}
%\INSconfig{}{SOFI}{FastPhot}{Provide list of filters HERE}
%\INSconfig{}{SOFI}{Polarimetry}{Provide list of filters HERE}
%\INSconfig{}{SOFI}{Spectroscopy-long-slit}{Blue Grism, Provide list of slits HERE}
%\INSconfig{}{SOFI}{Spectroscopy-long-slit}{Red Grism, Provide list of slits HERE}
%\INSconfig{}{SOFI}{Spectroscopy-high-res}{H, Provide list of slits HERE}
%\INSconfig{}{SOFI}{Spectroscopy-high-res}{K, Provide list of slits HERE}
%
%
%-----------------------------------------------------------------------
%---- HARPS at the 3.6 -------------------------------------------------
%-----------------------------------------------------------------------
%
%\INSconfig{}{HARPS}{spectro-Thosimult}{HARPS}
%\INSconfig{}{HARPS}{WAVE}{HARPS}
%\INSconfig{}{HARPS}{spectro-ObjA(B)}{HARPS}
%\INSconfig{}{HARPS}{spectro-ObjA(B)}{EGGS}
%\INSconfig{}{HARPS}{spectro-polarimetry}{linear}
%\INSconfig{}{HARPS}{spectro-polarimetry}{circular}
%
%
%%%%%%%%%%%%%%%%%%%%%%%%%%%%%%%%%%%%%%%%%%%%%%%%%%%%%%%%%%%%%%%%%%%%%%%%
% Chajnantor
%-----------------------------------------------------------------------
%---- SHFI at APEX ----------------------------------------------
%-----------------------------------------------------------------------
%
%\INSconfig{}{SHFI}{APEX-1}{Please enter Central Frequency 211 to 275 GHz}
%\INSconfig{}{SHFI}{APEX-2}{Please enter Central Frequency 275 to 370 GHz}
%\INSconfig{}{SHFI}{APEX-3}{Please enter Central Frequency 385 to 500 GHz} 
%
%-----------------------------------------------------------------------
%---- LABOCA at APEX ----------------------------------------------
%-----------------------------------------------------------------------
%
%\INSconfig{}{LABOCA}{IMG}{-}
%\INSconfig{}{LABOCA}{PHOT}{-}
%
%-----------------------------------------------------------------------
%---- Artemis at APEX ----------------------------------------------
%-----------------------------------------------------------------------
%
%\INSconfig{}{ARTEMIS}{IMG}{350 um}
%
%-----------------------------------------------------------------------
%---- FLASH at APEX ----------------------------------------------
%-----------------------------------------------------------------------
%
%\INSconfig{}{FLASH}{-}{Please enter Central Frequency 272 to 377 GHz and 385 to 495 GHz}
%
%-----------------------------------------------------------------------
%-----------------------------------------------------------------------
%---- SEPIA at APEX ----------------------------------------------
%-----------------------------------------------------------------------
%
%\INSconfig{}{SEPIA}{Band-5}{Please enter Central Frequency 159 to 211 GHz}
%




%%%%%%%%%%%%%%%%%%%%%%%%%%%%%%%%%%%%%%%%%%%%%%%%%%%%%%%%%%%%%%%%%%%%%%%%
%%%%% Interferometry PAGE %%%%%%%%%%%%%%%%%%%%%%%%%%%%%%%%%%%%%%%%%%%%%%
%%%%%%%%%%%%%%%%%%%%%%%%%%%%%%%%%%%%%%%%%%%%%%%%%%%%%%%%%%%%%%%%%%%%%%%%
%
% The \VLTITarget macro is only needed when requesting
% Interferometry, in which case it is MANDATORY to uncomment it and
% fill in the information. It takes the following parameters:
%
% 1st argument: run ID
% Valid values: run IDs specified in BOX 3
%
% 2nd argument: target name
% This parameter is NOT checked at the pdfLaTeX compilation.
%
% 3rd argument: visual magnitude
% Values with up to decimal places are allowed here.
% This parameter is NOT checked at the pdfLaTeX compilation.
%
% 4th argument: magnitude at wavelength of observation
% Values with up to decimal places are allowed here.
% This parameter is NOT checked at the pdfLaTeX compilation.
%
% 5th argument: wavelength of observation (in microns)
% Values with up to decimal places are allowed here.
% This parameter is NOT checked at the pdfLaTeX compilation.
%
% 6th argument: size at wavelength of observation (in mas)
% This parameter is NOT checked at the pdfLaTeX compilation.
%
% 7th argument: baseline
% UT observations are scheduled in terms of 3-telescope
% baselines for AMBER and 4-telescope baselines for PIONIER.
% For UT observations please specify one of the available
% AMBER or PIONIER baselines.
%
% AT observations are scheduled in terms of 4-telescope
% configurations (quadruplets). For these observations, the
% time can be split among the different baselines; the exact
% baselines will be specified at Phase 2.
% For AT observations, please specify only one of the 3
% available AT quadruplets at this stage.
%
%
% 8th parameter: Range of visibilities for the specified configuration.
% Please specify the maximum and minimum visibility values
% corresponding to the chosen configuration at hour angle 0
% separated by "/".
% This parameter is NOT checked at the pdfLaTeX compilation. 
%
% 9th parameter: correlated magnitude
% (for the visibility values specified in the 8th parameter)
% This parameter is NOT checked at the pdfLaTeX compilation.
%
%
% 10th parameter: time on target in hours
% Values with up to decimal places are allowed here.
% This parameter is NOT checked at the pdfLaTeX compilation.
%
% The available baselines for Period 96 are shown below.
% For AT observations, the time can be split at Phase 2 among the
% different offered baselines of the chosen quadruplet. 
% Please see the Call for Proposals for more information.
%
%
% 
% AMBER
% A0-B2-C1-D0
% A0-J2-G1-J3
% D0-K0-G2-J3
% UT1-UT2-UT3
% UT1-UT2-UT4
% UT1-UT3-UT4
% UT2-UT3-UT4
% 
% PIONIER
% A0-B2-C1-D0
% A0-J2-G1-J3
% D0-K0-G2-J3
% UT1-UT2-UT3-UT4
% 

%\VLTITarget{E}{Alpha Ori}{-1.4}{-1.4}{10.6}{6}{UT1-UT2-UT3}{0.60/0.10}{-0.2/4.0}{2} 
%\VLTITarget{F}{Alpha Ori}{-1.4}{-1.4}{10.6}{6}{D0-K0-G2-J3}{0.80/0.40}{-0.9/-0.2}{1} 

% You can specify here a note applying to all or some of your VLTI
% targets.  You should take advantage of this note to indicate
% suitable alternative baselines for your observations.
% This macro is NOT checked at the pdfLaTeX compilation.

%\VLTITargetNotes{Note about the VLTI targets, e.g., Run E can also be carried out using UT1-UT3-UT4.}

%%%%%%%%%%%%%%%%%%%%%%%%%%%%%%%%%%%%%%%%%%%%%%%%%%%%%%%%%%%%%%%%%%%%%%%%
%%%%% ToO PAGE %%%%%%%%%%%%%%%%%%%%%%%%%%%%%%%%%%%%%%%%%%%%%%%%%%%%%%%%%
%%%%%%%%%%%%%%%%%%%%%%%%%%%%%%%%%%%%%%%%%%%%%%%%%%%%%%%%%%%%%%%%%%%%%%%%
%
% The \ToOrun macro is needed only when requesting Target of
% Opportunity (ToO) observations, in which case it is MANDATORY to
% uncomment it and fill in the information. It takes the following
% parameters: 
%
% 1st argument: run ID
% Valid values: run IDs specified in BOX 3
%
% 2nd argument: nature of observation
% Valid values: RRM, ToO-hard, ToO-soft
%
% 3rd argument: number of targets per run
% This parameter is NOT checked at the pdfLaTeX compilation.
%
% 4th argument: number of triggers per targets
% (for RRM and ToO observations only)
% This parameter is NOT checked at the pdfLaTeX compilation.

\TOORun{A}{ToO-soft}{2}{2}
%\TOORun{A}{RRM}{2}{3}
%\TOORun{B}{ToO-hard}{3}{1}

% You have the opportunity to add notes to the ToO runs by using
% the \TOONotes macro.
% This macro is NOT checked at the pdfLaTeX compilation.

%\TOONotes{Use this macro to add a note to the ToO page.}


%%%%%%%%%%%%%%%%%%%%%%%%%%%%%%%%%%%%%%%%%%%%%%%%%%%%%%%%%%%%%%%%%%%%%%%%
%%%%% VISITOR SPECIAL INSTRUMENT PAGE %%%%%%%%%%%%%%%%%%%%%%%%%%%%%%%%%%
%%%%%%%%%%%%%%%%%%%%%%%%%%%%%%%%%%%%%%%%%%%%%%%%%%%%%%%%%%%%%%%%%%%%%%%%
%
% The following commands are only needed when bringing a Visitor
% Special Instrument, in which case it is MANDATORY to uncomment them
% and provide all the required information.
%
%\Desc{}   %Description of the instrument and its operation
%\Comm{}   %On which telescope(s) has instrument been commissioned/used
%\WV{}     %Total weight and value of equipment to be shipped
%\Wfocus{} %Weight at the focus (including ancillary equipment)
%\Interf{} %Compatibility of attachment interface with required focus
%\Focal{}  %Back focal distance value
%\Acqu{}   %Acquisition, focusing, and guiding procedure
%\Softw{}  %Compatibility with ESO software standards (data handling)
%\Suppl{}  %Estimate of services expected from ESO (in person days)

%%%%%%%%%%%%%%%%%%%%%%%%%%%%%%%%%%%%%%%%%%%%%%%%%%%%%%%%%%%%%%%%%%%%%%%%
%%%%% THE END %%%%%%%%%%%%%%%%%%%%%%%%%%%%%%%%%%%%%%%%%%%%%%%%%%%%%%%%%%
%%%%%%%%%%%%%%%%%%%%%%%%%%%%%%%%%%%%%%%%%%%%%%%%%%%%%%%%%%%%%%%%%%%%%%%%
\MakeProposal
\end{document}


